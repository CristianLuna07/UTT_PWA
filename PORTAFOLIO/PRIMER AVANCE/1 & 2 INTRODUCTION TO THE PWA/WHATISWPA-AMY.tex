\documentclass{article}
\usepackage{enumitem}
\usepackage{graphicx}
\usepackage[letterpaper, margin=1in]{geometry}
\usepackage{times}
\usepackage{newtxtext}
\usepackage{multicol}  % Add this line for the multicol package
\usepackage{natbib}  
\usepackage{hyperref} 
\usepackage{float}
\usepackage{sectsty}
\usepackage{url} % Add this line for the url package
\usepackage{bookmark} % Add this line for the bookmark package

\sectionfont{\fontsize{11}{15}\selectfont}
\subsectionfont{\fontsize{11}{14.4}\selectfont}  % Adjust the font size as needed
\subsubsectionfont{\fontsize{11}{13.2}\selectfont} 

\begin{document}
	\begin{center}
		\includegraphics[width=0.6\textwidth]{img/UTT.png}
		\vspace{2cm}

		{\fontsize{24pt}{28.8pt}\selectfont \fontfamily{ptm}\selectfont \textbf{Optimizing Home Energy Management: The Transformative Role of Progressive Web Apps (PWA) in Residential Electrical Measurements}}
		\vspace{1cm}

		{\Large Made by:}

		\vspace{0.5cm}
		{\fontsize{12pt}{28.8pt}\selectfont Luna Perez Cristian}
		
		{\large January 17th, 2024}
	\end{center}
	\begin{multicols}{2}
	\section{Progressive Web Applications (PWA)}
	\textbf{Progressive Web Applications (PWA)} are a combination of the best of web applications and native applications. They are commonly defined as apps that bring together the best of web and native applications. PWAs automatically adapt to the device's screen size and can leverage mobile device features, such as the camera or location. Some of the main features of PWAs include:
	
	\begin{itemize}
		\item \textbf{Responsive}: They automatically adapt to any format, browser, or device.
		\item \textbf{Updated}: PWAs always show their latest version to the user, using automatic updates.
		\item \textbf{Secure}: They always use the secure HTTPS protocol.
		\item \textbf{Fast}: A PWA has optimized speed, both for loading and navigation.
	\end{itemize}
	
	\section{Differences between Applications}
	\begin{itemize}
		\item \textbf{Mobile applications}: Also called native applications, they are only available for smartphones, tablets, and other small touch devices. To install them, you must access an app store, such as Google Play (Android), Windows Store (Windows), or App Store (iOS).
		\item \textbf{Web applications}: They run in a web browser. A web version can be developed to use them on all types of screens.
		\item \textbf{Progressive web applications (PWA)}: They are web applications that use emerging APIs and browser features along with a traditional progressive enhancement strategy to offer a native application.
	\end{itemize}
	
	\section{Advantages and Disadvantages}
	\begin{itemize}
		\item \textbf{Advantages of PWAs}: They are secure apps, work with (almost) any browser, are responsive, work offline, are permanently updated, can be found through search engines, can be linked through a URL, can be "installed" by anchoring them to your mobile's desktop.
		\item \textbf{Disadvantages of PWAs}: Their battery consumption is higher than that produced in native applications and, in addition, they cannot be downloaded in common app stores.
		\item \textbf{Advantages of web applications}: They are a quick and cheap solution, as the investment destined for their development is lower and less time is needed.
	\end{itemize}
	\section{Tools for PWA Development and Execution}
	There are several tools and frameworks available for the development and execution of Progressive Web Apps (PWAs). Some of the most popular ones include:
	\begin{itemize}
		\item \textbf{ReactJS}: A JavaScript library for building user interfaces, maintained by Facebook.
		\item \textbf{AngularJS}: A JavaScript-based open-source front-end web application framework mainly maintained by Google.
		\item \textbf{VueJS}: A JavaScript framework for building user interfaces.
		\item \textbf{Ionic}: An open-source SDK for hybrid mobile app development.
		\item \textbf{Polymer}: An open-source JavaScript library for building web applications using Web Components.
	\end{itemize}
	
	\section{Objectives of PWAs}
	The main objectives of PWAs are:
	\begin{itemize}
		\item To provide a user experience on par with native mobile applications.
		\item To increase speed and cost of development as compared to by-device native development.
		\item To provide connectivity flexibility: work offline or on low-quality networks.
		\item To eliminate app installation friction, avoid app stores’ “walled garden”.
		\item To eliminate version release and support headaches.
	\end{itemize}
	
	\section{Requirements of PWAs}
	For a web app to be considered a PWA, it must meet the following requirements:
	\begin{itemize}
		\item The website must be served from a secure (HTTPS) domain.
		\item It must have a web app manifest, a JSON file that provides information about the application (such as name, author, icon, and description) in a text file.
		\item It must have a service worker registered, to allow the app to work offline.
		\item The web app is not already installed.
		\item Meets the user engagement heuristics: The user needs to have clicked or tapped on the page at least once.
	\end{itemize}
	
	\section{Types of Tools for a PWA Development Environment}
	There are several types of tools that can be used in a PWA development environment:
	\begin{itemize}
		\item \textbf{Front-end PWA Development Tools}: These tools are used to design the user interface of the PWA. Examples include ReactJS, AngularJS, and VueJS.
		\item \textbf{Backend PWA Development Tools}: These tools are used to manage the server-side operations of the PWA.
		\item \textbf{Offline Capabilities and Data Synchronization Tools}: These tools are used to enable the PWA to function offline and synchronize data when the connection is restored.
		\item \textbf{Testing and Debugging PWA Development Tools}: These tools are used to test the PWA for bugs and errors during the development process.
	\end{itemize}
	\section{Referencias}
	\begin{itemize}
		\item Kremer, M. (2023, 10 de marzo). What are the key benefits and features of a PWA? Customer Engagement Blog. Recuperado de \url{https://onesignal.com/blog/what-is-a-pwa/}
		\item MDN Web Docs. (2023, 4 de julio). What is a progressive web app? - Progressive Web apps. Recuperado de \url{https://developer.mozilla.org/en-US/docs/Web/Progressive_web_apps/Guides/What_is_a_progressive_web_app/}
		\item Singh, H. (2023, 30 de noviembre). 10 Best Progressive Web App (PWA) Frameworks in 2023. The NineHertz. Recuperado de \url{https://theninehertz.com/blog/progressive-web-apps-frameworks}
		\item Kvartalnyi, N., \& Kvartalnyi, N. (2023, 16 de enero). Benefits of Progressive Web Apps (PWA) – Advantages and Disadvantages. Inoxoft. Recuperado de \url{https://inoxoft.com/blog/benefits-of-progressive-web-apps\\-pwa-advantages-and-disadvantages/}
		\item Husar, A. (2023, 1 de mayo). The best tools for building progressive web apps. Stackify. Recuperado de \url{https://stackify.com/the-best-tools-for-building-progressive\\-web-apps/}
		\item AppsChopper. (2021, 21 de diciembre). 9 Best tools and Technologies to leverage for Progressive Web App (PWA) development. AppsChopper Blog. Recuperado de \url{https://www.appschopper.com/blog/best-tools-technologies-leverage-progre\\ssive-web-app-pwa-development/}
	\end{itemize}
	\end{multicols}
\end{document} 

